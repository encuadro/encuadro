\chapter{Rendering}
\section{Introducci�n}
\textit{Rendering} es un t�rmino en ingl�s que denota el proceso de generar una imagen 2D a partir de un modelo digital 3D o un conjunto de ellos, a los que se les llama ``escena''. Puede ser comparado a tomar una foto o filmar una escena en la vida real.

\section{ISGL3D}
ISGL3D es un \textit{framework} (marco de trabajo) para \textit{iPad}, \textit{iPhone} y \textit{iPod touch} escrito en \textit{Objective-C}, que sirve para crear escenas y \textit{renderizarlas} de forma sencilla. Es un proyecto en c�digo abierto y gratis. En su sitio web oficial: \url{www.isgl3d.com}, se puede descargar el c�digo de ISGL3D y de forma sencilla este puede ser agregado como un complemento de \textit{Xcode}. Adem�s se pueden encontrar tutoriales, una \textit{Application Programming Interface} (API) y un acceso a un grupo de \textit{google} donde la comunidad pregunta y responde dudas propias y ajenas. \\

Cuando se crea una aplicaci�n ISGL3D, el n�cleo de la misma es la llamada ``\textit{view}'' (``vista'' en Espa\~nol ). Una \textit{view} esta compuesta por una escena y una c�mara:
\begin{itemize}
\item Una \textbf{escena} (Isgl3dScene3D) a donde los objetos o modelos 3D son agregados como nodos. Todos los nodos pueden ser tanto trasladados como rotados y pueden tener otros nodos hijos; los nodos hijos son trasladados y rotados con sus padres. As� como objetos 3D, se pueden agregar luces de distinto tipo, que generar�n en la escena efectos de sombra que luego ser�n adecuadamente \textit{renderizados} en funci�n de d�nde se encuentre y hacia d�nde este mirando la c�mara.
\item Una \textbf{c�mara} es utilizada para para visualizar la escena desde una posici�n y un �ngulo en particular. La c�mara se manipula como cualquier otro objeto o nodo en la escena, se puede trasladar, rotar y hasta indicar hacia d�nde quiere uno que la c�mara apunte. Es importante ajustar la c�mara de manera que su arquitectura sea la que uno busca. Se puede entonces ajustar ciertos par�metros intr�nsecos a esta como por ejemplo su campo visual, su distancia focal, su altura y anchura, etc.
\end{itemize}
Es importante entender que el llamado \textit{render} se realiza sumando la informaci�n de la escena, objetos 3D y sus hijos, luces, etc.; m�s la informaci�n de d�nde se encuentra la c�mara, sus caracter�sticas y hacia d�nde esta apunta.