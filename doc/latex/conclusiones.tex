\chapter{Conlusiones y trabajo a futuro}

\section{Introducci\'on}

En el presente cap�tulo se analiza al proyecto en su conjunto, primero desde el cumplimiento o no de sus objetivos y luego se har� una discusi�n t�cnica respecto de las elecciones realizadas, sobre todo en cuanto a los algoritmos utilizados para implementar cada uno de los bloques involucrados en la realidad aumentada. Finalmente, se plantea el trabajo que se podr\'ia hacer a futuro de manera de mejorar la aplicaci\'on final.

\section{Conslusiones generales}

En l\'ineas generales, se lo puede decir que el presente proyecto result� exitoso. En primer lugar, porque se cree que se cumplieron con todos los objetivos originales. Al d�a de hoy se cuenta con una documentaci�n de toda la investigaci�n realizada que brinda herramientas y un registro de la experiencia a todo aquel que busque continuar con esta l�nea de investigaci�n. Adem�s, se logr� implementar varios casos de uso que solucionan diferentes problemas en los que se puede emplear realidad aumentada. Finalmente, se implement� el mencionado recorrido interactivo por el museo, que cuanta con los bloques de \textit{navegaci�n}, \textit{reconocimiento de obra} y \textit{realidad aumentada} que se explican en el cap�tulo \ref{chap: alcance}.\\

En segundo lugar, se destaca el hecho de haber logrado desarrollar un aplicaci�n completa que funcione correctamente. Se cree que el desempe\~no de la misma es muy bueno, y en particular, si bien la realidad aumentada puede mejorar, conforma.\\

En cuanto al funcionamiento del grupo de trabajo, se cree que fue excelente. Se adquiri� una din�mica de subgrupos de trabajo que resultaron muy eficientes. Se respet� mucho la divisi�n de tareas y el trabajo se mantuvo perfectamente equilibrado, con integrantes proactivos en todo momento.\\

Finalmente, la relaci�n con el tutor fue muy buena. Se cree que fue un excelente orientador y que su disposici�n al proyecto fue notable.\\

\section{Conslusiones espec\'ificas}

\section{Trabajo a futuro}