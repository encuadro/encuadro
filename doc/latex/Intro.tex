\chapter{Introducci�n}

\section{Sobre el documento}
\label{sec:Sobre el documento}
Este trabajo cuenta con la documentaci�n del Proyecto \textit{enCuadro}, en el marco del proyecto de fin de carrera, para la fecha del segundo hito. En el mismo, se encuentra una comparaci�n entre lo que se planteaba tener para la fecha durante la planificaci�n y lo que se tiene realmente ahora; adem�s de un cronograma con las actividades a desempe�ar a lo largo de los meses que quedan para poder cumplir con los objetivos.\\
Es conveniente recordar al lector que este proyecto cuenta con tres partes fundamentales:
\begin{itemize}
\item Navegaci�n
\item Identificaci�n de obras
\item Realidad aumentada
\end{itemize}

Si bien la mayor parte de las energ�as se han enfocado en la realidad aumentada, las dos primeras partes son importantes en lo que refiere a la completitud del desarrollo de la aplicaci�n.\\
En este documento se hace tambi�n una descripci�n detallada de las conclusiones obtenidas hasta el momento en cuanto al sistema de navegaci�n que se eligi�, los algoritmos para identificar las obras que se han estudiado, las herramientas de \textit{rendering} utilizadas para la realidad aumentada y los algoritmos de estimaci�n de pose que se han implementado y comparado. Se presentan adem�s los casos de uso que aspiramos llegar a implementar.\\

\section{Entregables planificados para la fecha del segundo hito}
\label{sec:Entregables planificados para la fecha del segudo hito}
A continuaci�n se listan los trabajos que se esperaba tener prontos cuando se realiz� la planificaci�n, en octubre de 2011.\\
 
\begin{itemize}
\item Implementaci�n del algoritmo de navegaci�n
\item Implementaci�n del algoritmo de realidad aumentada
\item Implementaci�n del algoritmo de identificaci�n
\item Concatenaci�n de los tres bloques anteriores
\item Prueba del programa
\end{itemize}

A lo largo de este documento se detallar� el estado de cada uno de ellos.

